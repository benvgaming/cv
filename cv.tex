\documentclass{res}
\usepackage{helvetica} % uses helvetica postscript font (download helvetica.sty)
\newsectionwidth{0pt}  % So the text is not indented under section headings
\usepackage{fancyhdr}  % use this package to get a 2 line header
\usepackage[utf8]{inputenc}
\renewcommand{\headrulewidth}{0pt} % suppress line drawn by default by fancyhdr
\setlength{\headheight}{24pt} % allow room for 2-line header
\setlength{\headsep}{24pt}  % space between header and text
\setlength{\headheight}{24pt} % allow room for 2-line header
\pagestyle{fancy}
\rhead{ {\it Vincent Fournier}\\{\it p. \thepage} }
\cfoot{}
\topmargin=-0.5in

\begin{document}
\thispagestyle{empty} % this page has no header

{\bf\huge Vincent Fournier}
\hfill vince@ntfournier.com \\
Étudiant en génie logiciel à l’École de technologie supérieure.
\hfill 514-949-8462 \\
...  \hfill \textbf {github}: ntfournier

\begin{resume}

% --------------------------------
\section{Formation}
\vspace{8pt}

{\sl Baccalauréat en génie logiciel}, \\
École de technologie supérieure (ÉTS), Montréal
\hfill Depuis septembre 2013


{\sl Technique en informatique de gestion}, \\
Cégep de Saint-Jean-sur-Richelieu
\hfill 2010-2013

\vspace{0.1in}

% --------------------------------
\section{Experiences professionelles}
\vspace{8pt}

{\sl Savoir-faire Linux, Montréal}
\hfill mai à septembre 2015 \\
Développeur \hfill (Stage universitaire 2)

\begin{itemize} \itemsep -2pt % reduce space between items
	\item Développement d'une solution Python de "Monitoring as a Service" pour Openstack;
	\item Maintient et développement d'une application AngularJs avec des vues configurables;
	\item Création de paquets RPM (RedHat) pour l'installation de la solution;
	\item Participation à une processus agile.
\end{itemize}


{\sl Jabez Technologies, Montréal}
\hfill janvier à avril 2014 \\
Développeur web \hfill (Stage universitaire 1)

\begin{itemize} \itemsep -2pt
	\item Conversion d’un site web en PHP vers une application web moderne en JavaScript avec NodeJs et AngularJs;
	\item Implémentation de plusieurs fonctionnalités telles qu’un système conventionnel de gestion d’utilisateurs avec des rôles et des droits;
	\item Gestion de l’ensemble du projet de manière Agile avec des tâches et un "board".
\end{itemize}


{\sl Université Catholique de Lyon, Lyon, France}
\hfill avril à mai 2013 \\
Développeur au service informatique \hfill (Stage collégial)

\begin{itemize} \itemsep -2pt
	\item Création d’une application avec l’aide d’un ETL qui extrait les données d’une base de données, les modifient et les acheminent à la base de données du système d’éducation public français.
	\item Analyse et remplacement d’un logiciel désuet dans le but de créer une application plus conforme et extensible.
\end{itemize}


{\sl Résidence du Carrefour, Saint-Jean-sur-Richelieu}
\hfill septembre à mars 2013 \\
Chef d'équipe \hfill (Projet de fin d'études collégials)

\begin{itemize} \itemsep -2pt
	\item Création d’une application web avec Java et JSP pour un client.
	\item Conduire des sessions de travail avec le client pour définir les besoins et analyser les voies de solutions.
	\item Présenter l’offre finale au client et traiter les demandes de changements.
\end{itemize}

\vspace{0.1in}


% --------------------------------
\section{Connaissances informatique}
\vspace{12pt}

Langages de programmation
\begin{itemize}
	\item Java, JavaScript, Python.
\end{itemize}

Logiciels maîtrisés
\begin{itemize}
	\item GNU/Linux, IntelliJ Idea (WebStorm, PyCharm), Vim.
\end{itemize}

\newpage

% --------------------------------
\section{Projets accomplis}
\vspace{8pt}

{\sl Membre du club Formule ÉTS} \hfill (Club étudiant)
\begin{itemize} \itemsep -2pt
	\item Création d’un document de vision ainsi qu’un document de spécification des requis pour une application de télémétrie en temps réel;
	\item Prototypage de l’application en JavaScript
\end{itemize}

{\sl Membre du club Conjure} \hfill (Club étudiant)
\begin{itemize} \itemsep -2pt
	\item Création d’un jeu en C\# avec le logiciel Unity.
	\item Gestion du projet.
\end{itemize}

{\sl Création de contenus multimédia} \hfill (Projets personnel)
\begin{itemize} \itemsep -2pt
	\item Développement d’un jeu Android en Java avec la librairie LibGDX.
	\item Création d’un clone de Geometry Wars en C\# avec le cadriciel XNA.
	\item Publication d’une application Windows Phone en C\#.
\end{itemize}

\vspace{0.1in}


% --------------------------------
\section{Expériences connexes}
\vspace{12pt}

{\sl Chef d'équipe, secteur restauration}
	\hfill mai à août 2011, 2012 et 2013 \\
Parc d'attraction La Ronde

\begin{itemize} \itemsep -2pt
	\item Gestion et formation de 8 à 11 personnes.
	\item Gestion des inventaires, des commandes et du fonctionnement de quatre points de vente.
	\item Responsable d’une partie de la communication entre les superviseurs et les préposés.
	\item Formation les nouveaux employés.
\end{itemize}


{\sl Cuisinier, secteur restauration}
	\hfill mai à août 2010 \\
Parc d'attraction La Ronde

\begin{itemize} \itemsep -2pt
	\item Gestion des périodes d’achalandages et de mon espace de travail.
\end{itemize}

\vspace{0.1in}


% --------------------------------
\section{Réalisations}
\vspace{12pt}

\begin{itemize} \itemsep -2pt
	\item Promotion du programme collégial de technique informatique;
	\item Choisi par le coordonnateur du programme pour faire une entrevue avec le journal régional pour la promotion du programme collégial;
	\item Participation aux portes ouvertes pour la présentation des projets accomplis durant mon parcours.
\end{itemize}

\end{resume}
\end{document}
