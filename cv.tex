\documentclass{res}
\usepackage{helvetica} % uses helvetica postscript font (download helvetica.sty)
\newsectionwidth{0pt}  % So the text is not indented under section headings
\usepackage{fancyhdr}  % use this package to get a 2 line header
\usepackage[utf8]{inputenc}
\renewcommand{\headrulewidth}{0pt} % suppress line drawn by default by fancyhdr
\newcommand{\inFrench}[1]{#1}
\newcommand{\inEnglish}[1]{}
\setlength{\headheight}{24pt} % allow room for 2-line header
\setlength{\headsep}{16pt}  % space between header and text
\setlength{\headheight}{24pt} % allow room for 2-line header
\pagestyle{fancy}
\rhead{ {\it Vincent Fournier}\\{\it p. \thepage} }
\cfoot{}
\topmargin=-0.7in

\begin{document}
\thispagestyle{empty} % this page has no header

{\bf\huge Vincent Fournier}
\inFrench{
	\hfill \textbf {github.com/ntfournier} \\
	Étudiant en génie logiciel à l’École de technologie supérieure.
	\hfill \textbf {vince@ntfournier.com} \\
	Disponible pour stage universitaire 3 en \textbf{mai 2016} pour une durée 4 ou 8 mois.
	\hfill \textbf {tel}: 514-949-8462 \\
}
\inEnglish{
	\hfill \textbf {github.com/ntfournier} \\
	Software engineering student at l'École de technologie supérieure.
	\hfill \textbf {vince@ntfournier.com} \\
	Available for third university internship in \textbf{May 2016} for a length 4 or 8 months.
	\hfill \textbf {tel}: 514-949-8462 \\
}
\vspace{-15pt}

\begin{resume}
% --------------------------------
\section{Formation}
\vspace{6pt}

\inFrench{
	% Bacc
	École de technologie supérieure (ÉTS), Montréal
	\hfill Depuis septembre 2013 \\
	{\sl Baccalauréat en génie logiciel}

	% Cegep
	Cégep de Saint-Jean-sur-Richelieu, Saint-Jean-sur-Richelieu
	\hfill 2010-2013 \\
	{\sl Technique en informatique de gestion}
}
\inEnglish{
	% Bacc
	École de technologie supérieure (ÉTS), Montreal
	\hfill Since September 2013 \\
	{\sl Bachelor in Software Engineering}

	% Cegep
	Cegep of Saint-Jean-sur-Richelieu, Saint-Jean-sur-Richelieu
	\hfill 2010-2013 \\
	{\sl Technique en informatique de gestion}
}


% --------------------------------
\inFrench{
	\section{Expériences professionnelles}
}
\inEnglish{
	\section{Professional experiences}
}
\vspace{6pt}

\inFrench{
	% SFL
	Savoir-faire Linux, Montréal
	\hfill mai à septembre 2015 \\
	{\sl Développeur} \hfill (Stage universitaire 2)
	\vspace{0.05in}

	\begin{itemize} \itemsep -2pt
		\item Développement d'une solution Python de "Monitoring as a Service" pour Openstack;
		\item Maintien et développement d'une application AngularJs avec des vues configurables;
		\item Création de paquets RPM (RedHat) pour l'installation de la solution;
		\item Participation à un processus agile.
	\end{itemize}

	% Robotmaster
	Jabez Technologies, Montréal
	\hfill janvier à avril 2014 \\
	{\sl Développeur web} \hfill (Stage universitaire 1)
	\vspace{0.05in}

	\begin{itemize} \itemsep -2pt
		\item Conversion d’un site web en PHP vers une application web moderne en JavaScript avec NodeJs et AngularJs;
		\item Implémentation de plusieurs fonctionnalités telles qu’un système conventionnel de gestion d’utilisateurs avec des rôles et des droits;
		\item Gestion de l’ensemble du projet de manière Agile avec des tâches et un "board".
	\end{itemize}

	% Lyon
	Université Catholique de Lyon, Lyon, France
	\hfill avril à mai 2013 \\
	{\sl Développeur au service informatique} \hfill (Stage collégial)
	\vspace{0.05in}

	\begin{itemize} \itemsep -2pt
		\item Création d’une application avec l’aide d’un ETL qui extrait les données d’une base de données, les modifie et les achemine à la base de données du système d’éducation public français;
		\item Analyse et remplacement d’un logiciel désuet dans le but de créer une application plus conforme et extensible.
	\end{itemize}

	% Carrefour
	Résidences du Carrefour, Saint-Jean-sur-Richelieu
	\hfill septembre à mars 2013 \\
	{\sl Chef d'équipe} \hfill (Projet de fin d'études collégiales)
	\vspace{0.05in}

	\begin{itemize} \itemsep -2pt
		\item Création d’une application web avec Java et JSP pour un client;
		\item Conduire des sessions de travail avec le client pour définir les besoins et analyser les voies de solutions;
		\item Présenter l’offre finale au client et traiter les demandes de changements.
	\end{itemize}
}

\inEnglish{
	% SFL
	Savoir-faire Linux, Montreal
	\hfill May to September 2015 \\
	{\sl Software developer} \hfill (Second University Internship)
	\vspace{0.05in}

	\begin{itemize} \itemsep -2pt
		\item Development of a solution of "Monitoring as a Service" for Openstack in Python.
		\item Maintaining and development of an AngularJs application with configurable views.
		\item Creation of RPM packets for installation of the solution on the Linux RedHat OS.
		\item Active participation in an agile process.
	\end{itemize}

	% Robotmaster
	Jabez Technologies, Montreal
	\hfill January to April 2014 \\
	{\sl Fullstack developer} \hfill (First University Internship)
	\vspace{0.05in}

	\begin{itemize} \itemsep -2pt
		\item Rewrite of a PHP website in a single page application in AngularJs and NodeJs.
		\item Creation of user and role base system.
		\item Agile management of the project.
	\end{itemize}

	% Lyon
	Catholic University of Lyon, Lyon, France
	\hfill April to May 2013 \\
	{\sl Software developer} \hfill (Collegial Internship)
	\vspace{0.05in}

	\begin{itemize} \itemsep -2pt
		\item Use of an ETL (Talend Open Studio) to extract data from a database to another.
		\item Replacement of old software for a scalable application.
	\end{itemize}

	% Carrefour
	Résidences du Carrefour, Saint-Jean-sur-Richelieu
	\hfill September to March 2013 \\
	{\sl Team leader} \hfill (End of formation project)
	\vspace{0.05in}

	\begin{itemize} \itemsep -2pt
		\item Meeting with client, elicitation of software requirements.
		\item Validation of the final offer and handle change requests.
		\item Creation of a Java Servlet application for the client.
	\end{itemize}
}


% --------------------------------
\inFrench{
	\section{Connaissances informatiques}
}
\inEnglish{
	\section{Software expertise}
}
\vspace{6pt}

\inFrench{
	Langages de programmation
	\vspace{0.05in}
	\begin{itemize}
		\item JavaScript (NodeJs, AngularJs), Python, Java.
	\end{itemize}

	Logiciels maîtrisés
	\vspace{0.05in}
	\begin{itemize}
		\item GNU/Linux, IntelliJ Idea (WebStorm, PyCharm), Vim.
	\end{itemize}
}

\inEnglish{
	Programming languages
	\vspace{0.05in}
	\begin{itemize}
		\item JavaScript (NodeJs, AngularJs), Python, Java.
	\end{itemize}

	Softwares
	\vspace{0.05in}
	\begin{itemize}
		\item GNU/Linux, IntelliJ Idea (WebStorm, PyCharm), Vim.
	\end{itemize}
}

\newpage

% --------------------------------
% Beginning of page 2
% --------------------------------
\inFrench{
	\section{Intérêts personnels}
}
\inEnglish{
	\section{Personal interests}
}
\vspace{6pt}

\inFrench{
	\vspace{6pt}
	\begin{itemize} \itemsep -2pt
		\item Développement Linux Kernel;
		\item Logiciels libres et open source;
		\item Nouvelles informatiques et avancées technologiques;
		\item Conception de jeux vidéos et ludification (gamification).
	\end{itemize}
}

\inEnglish{
	\vspace{6pt}
	\begin{itemize} \itemsep -2pt
		\item Linux Kernel development.
		\item Free software and open source.
		\item Computer science news and advances.
		\item Game design and gamification.
	\end{itemize}
}

% --------------------------------
\inFrench{
	\section{Implication scolaire et accomplissements}
}
\inEnglish{
	\section{Scholar involvement and accomplishment}
}
\vspace{6pt}

\inFrench{
	{\sl Membre du club Formule ÉTS} \hfill (Club étudiant)
	\vspace{0.05in}
	\begin{itemize} \itemsep -2pt
		\item Création d’un document de vision ainsi qu’un document de spécification des requis pour une application de télémétrie en temps réel;
		\item Prototypage de l’application en JavaScript.
	\end{itemize}

	{\sl Membre du club Conjure} \hfill (Club étudiant)
	\vspace{0.05in}
	\begin{itemize} \itemsep -2pt
		\item Création d’un jeu en C\# avec le logiciel Unity;
		\item Gestion du projet.
	\end{itemize}

	{\sl Promotion du programme collégial} \hfill (Cégep)
	\vspace{0.05in}
	\begin{itemize} \itemsep -2pt
		\item Entrevue avec le journal régional pour la promotion du programme collégial;
		\item Participation aux portes ouvertes pour la présentation des projets accomplis durant mon parcours.
	\end{itemize}

	{\sl Création de contenu multimédia} \hfill (Projets personnels)
	\vspace{0.05in}
	\begin{itemize} \itemsep -2pt
		\item Développement d’un jeu Android en Java avec la librairie LibGDX;
		\item Création d’un clone de Geometry Wars en C\# avec le cadriciel XNA;
		\item Publication d’une application Windows Phone en C\#.
	\end{itemize}
}

\inEnglish{
	{\sl Member of Formule ÉTS (racing car club)} \hfill (Student club)
	\vspace{0.05in}
	\begin{itemize} \itemsep -2pt
		\item Creation of a vision document and a software requirements document for an application with reel-time telemetry.
		\item Prototyping of the application in JavaScript.
	\end{itemize}

	{\sl Member of Conjure (video game creation club)} \hfill (Student club)
	\vspace{0.05in}
	\begin{itemize} \itemsep -2pt
		\item Creation of a game in C\# with Unity.
		\item Project management.
	\end{itemize}

	{\sl Cegep involvement} \hfill (Cegep)
	\vspace{0.05in}
	\begin{itemize} \itemsep -2pt
		\item Interview with regional newspaper for the promotion of my course of study.
		\item Presentation of the projects I have done in school to promote my program of study during the college open day.
	\end{itemize}

	{\sl Creation of multimedia content} \hfill (Personal projects)
	\vspace{0.05in}
	\begin{itemize} \itemsep -2pt
		\item Game development of an Android game in Java.
		\item Creation of a Geometry Wars clone in C\# with XNA framework.
		\item Release of a Windows Phone application in C\#.
	\end{itemize}
}


% --------------------------------
\inFrench{
	\section{Expériences connexes}
}
\inEnglish{
	\section{Other experiences}
}
\vspace{6pt}

\inFrench{
	{\sl Chef d'équipe, secteur restauration}
		\hfill mai à août 2011, 2012 et 2013 \\
	Parc d'attractions La Ronde

	\vspace{0.05in}
	\begin{itemize} \itemsep -2pt
		\item Gestion et formation de 8 à 11 personnes;
		\item Gestion des inventaires, des commandes et du fonctionnement de quatre points de vente;
		\item Responsable d’une partie de la communication entre les superviseurs et les préposés;
		\item Formation des nouveaux employés.
	\end{itemize}


	{\sl Cuisinier, secteur restauration}
		\hfill mai à août 2010 \\
	Parc d'attractions La Ronde

	\vspace{0.05in}
	\begin{itemize} \itemsep -2pt
		\item Gestion des périodes d’achalandages et de mon espace de travail.
	\end{itemize}
}

\inEnglish{
	{\sl Team leader, restauration sector}
		\hfill May to August 2011, 2012 and 2013 \\
	Amusement park, La Ronde

	\vspace{0.05in}
	\begin{itemize} \itemsep -2pt
		\item Management and formation of 8 to 11 employees.
		\item Inventory, command and sales management of four points of sale.
		\item Responsible of part of the communication between supervisors and employees.
	\end{itemize}


	{\sl Cook, restauration sector}
		\hfill May to August 2010 \\
	Amusement park, La Ronde

	\vspace{0.05in}
	\begin{itemize} \itemsep -2pt
		\item Management of the rush hours and of my work space.
	\end{itemize}
}

\vspace{0.1in}

\end{resume}
\end{document}
